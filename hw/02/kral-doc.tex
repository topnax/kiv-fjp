\documentclass[12pt, a4paper]{article}

\usepackage[czech]{babel}
\usepackage{lmodern}
\usepackage[utf8]{inputenc}
\usepackage[T1]{fontenc}
\usepackage{graphicx}
\usepackage{amsmath}
\usepackage[hidelinks,unicode]{hyperref}
\usepackage{float}
\usepackage{listings}
\usepackage{tikz}
\usepackage{xcolor}
\usepackage[final]{pdfpages}

\usepackage{syntax}
\usepackage{tikz}
\usepackage{tikz-qtree}

\definecolor{mauve}{rgb}{0.58,0,0.82}
\usetikzlibrary{shapes,positioning,matrix,arrows}

\newcommand{\img}[1]{(viz obr. \ref{#1})}

\definecolor{pblue}{rgb}{0.13,0.13,1}
\definecolor{pgreen}{rgb}{0,0.5,0}
\definecolor{pred}{rgb}{0.9,0,0}
\definecolor{pgrey}{rgb}{0.46,0.45,0.48}

\lstset{frame=tb,
  language=C,
  aboveskip=3mm,
  belowskip=3mm,
  showstringspaces=false,
  columns=flexible,
  basicstyle={\small\ttfamily},
  numbers=none,
  numberstyle=\tiny\color{gray},
  keywordstyle=\color{blue},
  commentstyle=\color{dkgreen},
  stringstyle=\color{mauve},
  breaklines=true,
  breakatwhitespace=true,
  tabsize=3
}

\lstset{language=Java,
  showspaces=false,
  showtabs=false,
  breaklines=true,
  showstringspaces=false,
  breakatwhitespace=true,
  commentstyle=\color{pgreen},
  keywordstyle=\color{pblue},
  stringstyle=\color{pred},
  basicstyle=\ttfamily,
  moredelim=[il][\textcolor{pgrey}]{$$},
  moredelim=[is][\textcolor{pgrey}]{\%\%}{\%\%}
}

\let\oldsection\section
\renewcommand\section{\clearpage\oldsection}

\begin{document}
	% this has to be placed here, after document has been created
	% \counterwithout{lstlisting}{chapter}
	\renewcommand{\lstlistingname}{Ukázka kódu}
	\renewcommand{\lstlistlistingname}{Seznam ukázek kódu}
    \begin{titlepage}

       \centering

       \vspace*{\baselineskip}

       \begin{figure}[H]
          \centering
          \includegraphics[width=7cm]{img/fav-logo.jpg}
       \end{figure}

       \vspace*{1\baselineskip}
        {\sc 2. domácí úkol z předmětu KIV/FJP}
       \vspace*{1\baselineskip}

       \vspace{0.75\baselineskip}

       {\LARGE\sc Využití bezkontextových gramatik\\}

       \vspace{4\baselineskip}
       
		\vspace{0.5\baselineskip}

       
       {\sc\Large Stanislav Král \\}

       \vspace{0.5\baselineskip}

       {A20N0091P}

       \vfill

       {\sc Západočeská univerzita v Plzni\\
       Fakulta aplikovaných věd}


    \end{titlepage}


    \tableofcontents
    \pagebreak


    
    \section{Zadání}

\begin{enumerate}
    \item Sestavte gramatiku popisující jazyk regulárních výrazů (s využitím operací zřetězení (bez operátorů, "\texttt{a}" a "\texttt{b}" se řetězí na "\texttt{ab}"), nebo ("\texttt{a + b}" - jedno nebo druhé, alternativně symbol "\texttt{|}" - v tomto případě pozor na odlišení terminálního symbolu a symbolu pro oddělení pravých stran v BNF), iterace ("\texttt{a*}" - libovolný počet opakování symbolu "\texttt{a}") a závorek které nastavují prioritu ("\texttt{(a + b)*}" je libovolná kombinace "\texttt{a}" a "\texttt{b}", "\texttt{a + b*}" je "\texttt{a}" následováno libovolným počtem "\texttt{b}").
Abyste se vyhnuli nutnosti pracovat s celou abecedou, použijte zástupný symbol "\texttt{a}" (nebo jiný) pro libovolný nevyhrazený znak ASCII.

    \item Demonstrujte fungování gramatiky na derivačních stromech alespoň 2 slov o délce alespoň 7 znaků. 

\end{enumerate}

    \section{Vypracování}

    \subsection{Sestavená gramatika}
    Při sestavování této gramatiky jsem vycházel ze cvičení ze dne 5.10.2020, kdy jsem rozšířil gramatiku popisující sčítání čísel o závorky, zřetězení a operátor iterace. 

    \begin{grammar}

        <E> ::= <E> `+' <T> | <ET> | <T>
        
        <T> ::= <B> | <F> | <E>

        <B> ::= `('<E>`)' | <a>

        <F> ::= <B>`*' | <T>
%        <E> ::= <E> `+' <T> | <T>
%        
%        <T> ::= <TF> | <F> | <F>`*'
%
%        <F> ::= <a> | `('<E>`)'
    \end{grammar}

\noindent Symbol "a" je zástupný symbol pro libovolný nevyhrazený znak ASCII. Symbol \texttt{<E>} je počátečním symbolem.

Pravidlem \texttt{F -> B*} zajišťuji, že operátor iterace lze použít pouze na závorku nebo jeden terminální symbol \texttt{a}.

\subsection{Ověření funkcionality}

\subsubsection{Derivační strom pro řetězec \texttt{(a + b)* ab*}}
\begin{tikzpicture}[sibling distance=45pt]
\tikzset{level distance=35pt}
\Tree [.E 
[.E [.T [.F [.B* [.(E) [.E+ [.T [.B [.a ] ] ] ] [.T [.B [.b ] ] ] ] ] ] ]  ] 
[.T [.E [.E [.F [.B [.a ] ] ] ] [.T [.F [.B* [.b ] ] ] ] ] ]
]
\end{tikzpicture}


\subsubsection{Derivační strom pro řetězec \texttt{ab*c + (ac)*}}
\begin{tikzpicture}[sibling distance=45pt]
\tikzset{level distance=35pt}
\Tree [.E
[.E+ [.E [.T [.B [.a ] ] ]  ] [.T [.E [.E [.T [.F [.B* [.b ] ] ] ] ] [.T [.B [.c ] ] ] ] ] ]
[.T [.F [.B* [.(E) [.E [.T [.B [.a ] ] ] ] [.T [.B [.c ] ] ] ] ] ] ]
 ]
\end{tikzpicture}

\end{document}    
