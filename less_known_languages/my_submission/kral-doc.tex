\documentclass[12pt, a4paper]{article}

\usepackage[czech]{babel}
\usepackage{lmodern}
\usepackage[utf8]{inputenc}
\usepackage[T1]{fontenc}
\usepackage{graphicx}
\usepackage{amsmath}
\usepackage[hidelinks,unicode]{hyperref}
\usepackage{float}
\usepackage{listings}
\usepackage{tikz}
\usepackage{xcolor}
\usepackage[final]{pdfpages}

\usepackage{syntax}
\usepackage{tikz}
\usepackage{tikz-qtree}

\definecolor{mauve}{rgb}{0.58,0,0.82}
\usetikzlibrary{shapes,positioning,matrix,arrows}

\newcommand{\img}[1]{(viz obr. \ref{#1})}

\definecolor{pblue}{rgb}{0.13,0.13,1}
\definecolor{pgreen}{rgb}{0,0.5,0}
\definecolor{pred}{rgb}{0.9,0,0}
\definecolor{pgrey}{rgb}{0.46,0.45,0.48}

\lstset{frame=tb,
  language=C,
  aboveskip=3mm,
  belowskip=3mm,
  showstringspaces=false,
  columns=flexible,
  basicstyle={\small\ttfamily},
  numbers=none,
  numberstyle=\tiny\color{gray},
  keywordstyle=\color{blue},
  commentstyle=\color{dkgreen},
  stringstyle=\color{mauve},
  breaklines=true,
  breakatwhitespace=true,
  tabsize=3
}

\lstset{language=Java,
  showspaces=false,
  showtabs=false,
  breaklines=true,
  showstringspaces=false,
  breakatwhitespace=true,
  commentstyle=\color{pgreen},
  keywordstyle=\color{pblue},
  stringstyle=\color{pred},
  basicstyle=\ttfamily,
  moredelim=[il][\textcolor{pgrey}]{$$},
  moredelim=[is][\textcolor{pgrey}]{\%\%}{\%\%}
}

\let\oldsection\section
\renewcommand\section{\clearpage\oldsection}

\begin{document}
	% this has to be placed here, after document has been created
	% \counterwithout{lstlisting}{chapter}
	\renewcommand{\lstlistingname}{Ukázka kódu}
	\renewcommand{\lstlistlistingname}{Seznam ukázek kódu}
    \begin{titlepage}

       \centering

       \vspace*{\baselineskip}

       \begin{figure}[H]
          \centering
          \includegraphics[width=7cm]{img/fav-logo.jpg}
       \end{figure}

       \vspace*{1\baselineskip}
        {\sc KIV/FJP - Workshop \#1}
       \vspace*{1\baselineskip}

       \vspace{0.75\baselineskip}

       {\LARGE\sc Méně známé jazyky - popis a analýza jazyka\\}

       \vspace{0.75\baselineskip}

       {\LARGE\sc Go\\}

       \vspace{4\baselineskip}
       
		\vspace{0.5\baselineskip}

       
       {\sc\Large Stanislav Král \\}

       \vspace{0.5\baselineskip}

       {A20N0091P}

       \vfill

       {\sc Západočeská univerzita v Plzni\\
       Fakulta aplikovaných věd}


    \end{titlepage}


    \tableofcontents
    \pagebreak
    
    \section{Vznik jazyka Go}

 V době, kdy softwarové služby a produkty společnosti Google začínají být využívány stále větším počtem uživatelů, používají backendoví vývojáři této společnosti v největší míře programovací jazyk C++. Mezi problémy, se kterými se musí denně potýkat, jsou dlouhé časy sestavení projektů a roustoucí nepřehlednost velkých projektů.

Koncem roku 2006 začíná tým vývojářů z Google, spolu s Kenem Thompsonem a Robertem Griesemerem, uvažovat nad vytvořením nového programovacího jazyka Go.  Hlavní myšlenkou není vytvořit jazyk s velkým množstvím nových funkcí a konstrukcí, ale vytvořit jazyk smysluplně konsolidující to nejlepší z ostatních populárních programovacích jazyků. Inspiraci hledají v jazycích Algol, C, C++, Newsqueak, Python nebo také například v jazyce Smalltalk, a řídí se pravidlem, že by Go měl být co nejjednodušší, a proto neobsahuje například následující konstrukce, které jsou v jiných jazycích zcela běžné:

\begin{itemize}
    \item hlavičkové soubory
    \item třídy
    \item dědičnost 
    \item konstruktory
    \item ukazatelovou aritmetiku
    \item výjimky
    \item globální proměnné
\end{itemize}

Vlastností, která byla při návrhu tohoto jazyka považována za velmi důležitou, byla schopnost jednoduše a elegantně definovat souběžné a paralelní programy. Velkou inspiraci a hlavní myšlenku autoři jazkya Go brali v unixových rourách.

V roce 2009 byl zdrojový kód jazyka Go uvolněn jako open-source, a v roce 2012 byla vydána verze 1.0. S tímto vydáním se začal klást důraz na využití a implementaci tohoto jazyka v projektech Google.

\section{Charakteristiky jazyka}

\subsection{Formální popis jazyka}
Gramatika popisující jazyk Go v rozšířené BNF formě je následující:

\begin{lstlisting}
Production  = production_name "=" [ Expression ] "." .
Expression  = Alternative { "|" Alternative } .
Alternative = Term { Term } .
Term        = production_name | token [ "..." token ] | Group | Option | Repetition .
Group       = "(" Expression ")" .
Option      = "[" Expression "]" .
Repetition  = "{" Expression "}" .

\end{lstlisting}

\end{document}    
