\documentclass[12pt, a4paper]{article}

\usepackage[czech]{babel}
\usepackage{lmodern}
\usepackage[utf8]{inputenc}
\usepackage[T1]{fontenc}
\usepackage{graphicx}
\usepackage{amsmath}
\usepackage[hidelinks,unicode,
            colorlinks = true,
            linkcolor = black,
            urlcolor  = blue
]{hyperref}
\usepackage{float}
\usepackage{listings}
\usepackage{listings-golang} 
\usepackage{tikz}
\usepackage{xcolor}
\usepackage[final]{pdfpages}

\usepackage{syntax}
\usepackage{tikz}
\usepackage{tikz-qtree}

\definecolor{mauve}{rgb}{0.58,0,0.82}
\usetikzlibrary{shapes,positioning,matrix,arrows}

\newcommand{\img}[1]{(viz obr. \ref{#1})}

\definecolor{pblue}{rgb}{0.13,0.13,1}
\definecolor{pgreen}{rgb}{0,0.5,0}
\definecolor{pred}{rgb}{0.9,0,0}
\definecolor{pgrey}{rgb}{0.46,0.45,0.48}

\lstset{frame=tb,
  language=C,
  aboveskip=3mm,
  belowskip=3mm,
  showstringspaces=false,
  columns=flexible,
  basicstyle={\small\ttfamily},
  numbers=none,
  numberstyle=\tiny\color{gray},
  keywordstyle=\color{blue},
  commentstyle=\color{dkgreen},
  stringstyle=\color{mauve},
  breaklines=true,
  breakatwhitespace=true,
  tabsize=3
}

\lstset{language=Java,
  showspaces=false,
  showtabs=false,
  breaklines=true,
  showstringspaces=false,
  breakatwhitespace=true,
  commentstyle=\color{pgreen},
  keywordstyle=\color{pblue},
  stringstyle=\color{pred},
  basicstyle=\ttfamily,
  moredelim=[il][\textcolor{pgrey}]{$$},
  moredelim=[is][\textcolor{pgrey}]{\%\%}{\%\%}
}

\let\oldsection\section
\renewcommand\section{\clearpage\oldsection}

\begin{document}
	% this has to be placed here, after document has been created
	% \counterwithout{lstlisting}{chapter}
	\renewcommand{\lstlistingname}{Ukázka kódu}
	\renewcommand{\lstlistlistingname}{Seznam ukázek kódu}
    \begin{titlepage}

       \centering

       \vspace*{\baselineskip}

       \begin{figure}[H]
          \centering
          \includegraphics[width=7cm]{img/fav-logo.jpg}
       \end{figure}

       \vspace*{1\baselineskip}
        {\sc KIV/FJP - Workshop \#1}
       \vspace*{1\baselineskip}

       \vspace{0.75\baselineskip}

       {\LARGE\sc Méně známé jazyky - popis a analýza jazyka\\}

       \vspace{0.75\baselineskip}

       {\LARGE\sc Go\\}

       \vspace{4\baselineskip}
       
		\vspace{0.5\baselineskip}

       
       {\sc\Large Stanislav Král \\}

       \vspace{0.5\baselineskip}

       {A20N0091P}

       \vfill

       {\sc Západočeská univerzita v Plzni\\
       Fakulta aplikovaných věd}


    \end{titlepage}


    \tableofcontents
    \pagebreak
    
    \section{Vznik jazyka Go}

 V době, kdy softwarové služby a produkty společnosti Google začínají být využívány stále větším počtem uživatelů, používají backendoví vývojáři této společnosti v největší míře programovací jazyk C++. Mezi problémy, se kterými se musí denně potýkat, jsou dlouhé časy sestavení projektů a roustoucí nepřehlednost velkých projektů.

Koncem roku 2006 začíná tým vývojářů z Google, spolu s Kenem Thompsonem a Robertem Griesemerem, uvažovat nad vytvořením nového programovacího jazyka Go.  Hlavní myšlenkou není vytvořit jazyk s velkým množstvím nových funkcí a konstrukcí, ale vytvořit jazyk smysluplně konsolidující to nejlepší z ostatních populárních programovacích jazyků. Inspiraci hledají v jazycích Algol, C, C++, Newsqueak, Python nebo také například v jazyce Smalltalk, a řídí se pravidlem, že by Go měl být co nejjednodušší, a proto neobsahuje například následující konstrukce, které jsou v jiných jazycích zcela běžné:

\begin{itemize}
    \item hlavičkové soubory
    \item třídy
    \item dědičnost 
    \item konstruktory
    \item ukazatelovou aritmetiku
    \item výjimky
    \item globální proměnné
\end{itemize}

Vlastností, která byla při návrhu tohoto jazyka považována za velmi důležitou, byla schopnost jednoduše a elegantně definovat souběžné a paralelní programy. Velkou inspiraci a hlavní myšlenku autoři jazkya Go brali v unixových rourách.

V roce 2009 byl zdrojový kód jazyka Go uvolněn jako open-source, a v roce 2012 byla vydána verze 1.0. S tímto vydáním se začal klást důraz na využití a implementaci tohoto jazyka v projektech Google. Nejnovější verze je 1.15 z 11.08.2020.

\section{Charakteristiky jazyka}

Při psaní této kapitoly bylo využíváno hlavně oficiální specifikace jazyka Go\footnote{\url{https://golang.org/ref/spec}} a webové stránky \url{https://gobyexample.com/}, kde se nachází užitečné ukázky a vysvětlení základních konstrukcí tohoto jazyka.

\subsection{Formální popis jazyka}
Gramatika popisující jazyk Go v rozšířené BNF formě je následující:

\begin{lstlisting}
Production  = production_name "=" [ Expression ] "." .
Expression  = Alternative { "|" Alternative } .
Alternative = Term { Term } .
Term        = production_name 
                | token [ "..." token ] 
                | Group 
                | Option 
                | Repetition .
Group       = "(" Expression ")" .
Option      = "[" Expression "]" .
Repetition  = "{" Expression "}" .

\end{lstlisting}

\subsection{Deklarace typu proměnné}
Explicitní deklarace typu proměnné není v jazyce Go nutná, jelikož překladač si typ proměnné doplní sám.


\begin{lstlisting}[caption={Typ proměnné nemusí být v kódu explicitně deklarován}, captionpos=b]
i := 42             // i is int
f := 3.142          // f is float64
g := 0.867 + 0.5i   // g is complex128
\end{lstlisting}

\noindent Pokud opravdu chceme typ explicitně deklarovat, tak je možné použít následující syntaxi:

\begin{lstlisting}[caption={Explicitní deklarace typu proměnné}, captionpos=b, language=Golang]
var i int = 0
var text string = "Hello Golang World!"
\end{lstlisting}


\subsection{Nepoužívané proměnné a balíčky}
Pokud deklarujeme nějakou proměnnou nebo importujeme balíček, který nepoužíváme, tak překlad programu skončí chybou. Pokud je proměnná pojmenována pouze znakem \texttt{_}, tak nemusí být používána nikdy nemusí, a program lze přeložit.


\subsection{Středníky}

V jazyce Go není ve většině případů nutné používat středníky. Středniky jsou automaticky lexerem doplněny při překladu.

\subsection{Řízení toku programu}

K řízení toku programu jsou v Go používány klíčová slova \texttt{if}, \texttt{for}, \texttt{switch}, \texttt{break} a \texttt{continue}.

Pro definování podmínky pro větvení programu obecně není nutné kulatých závorek kolem výrazu, který se má výkonat.

\begin{lstlisting}[caption={Větvení pomocí \texttt{if} a \texttt{switch}}, captionpos=b, language=Golang]
a := isEnabled()

if a {
    // do something
} else {
    // do something else
}

switch time.Now().Weekday() {
case time.Saturday, time.Sunday:
    fmt.Println("It's the weekend")
default:
    fmt.Println("It's a weekday")
}

\end{lstlisting}


\subsubsection{\texttt{for} jako \texttt{while}}
V Go chybí klíčové slovo \texttt{while}, místo kterého se používá \texttt{for}.
\begin{lstlisting}
for a < 10 {
    a++
    fmt.Println("a =", a)
}

\end{lstlisting}

\subsection{Typ \texttt{rune}}

Tento typ, který zabírá 32 bitů, slouží k uložení jednoho či více unicode znaků.

\subsection{Definice struktur a OOP}
V jazyce Go není podporován koncept tříd a dědičnosti, ale stejně jako například v jazyce C lze definovat struktury:

\begin{lstlisting}[caption={Definice struktur}, captionpos=b, language=Golang]
type person struct {
    name string
    age  int
}

p := person{name: "Foo", age: 50}
\end{lstlisting}

Avšak metody jsou implementovány jako funkce u kterých je kromě jména, parametrů a typů návratových hodnot, specifikován i takzvaný \textit{receiver} (příjemce), což je nějaký datový typ (nejčastěji struktura), nad kterým je tato metoda volána. Příjemcem může být samozřejmě i ukazatel na datový typ.

\begin{lstlisting}[caption={Volání metod nad strukturami}, captionpos=b, language=Golang]

func (p* person) haveBirthday() {
    p.age++
}

func (p person) isAdult() bool {
    return p.age >= 18
}

p := person{name: "Foo", age: 17}
p.isAdult()         // false
p.haveBirthday()    // increases the age to 18
p.isAdult()         // true
\end{lstlisting}

\subsection{Definice a implementace rozhraní}
Rozhraní v jazyku Go byla inspirována protokoly, s nimiž se lze setkat například v programovacím jazyku Smalltalk. Jedná se o specifikaci metod, které jsou společné pro entity s nějakou sdílenou vlastností nebo vlastnostmi.

\begin{lstlisting}[caption={Definice rozhraní}, captionpos=b, language=Golang]
type geometry interface {
    area() float64
    perim() float64
}
\end{lstlisting}

Na rozdíl od např. jazyku Java se v Go rozhraní implementuje implicitně. Pokud lze nad nějakou strukturou volat stejné metody, jako ty, které jsou definované v nějakém rozhraní, tak tato struktura automaticky dané rozhraní implementuje, a není tedy tak nutné explicitně danou strukturou označit např. klíčovým slovem \texttt{implements}.


\begin{lstlisting}[caption={Implementace rozhraní \texttt{geometry}}, captionpos=b, language=Golang]
func (r rect) area() float64 {
    return r.width * r.height
}
func (r rect) perim() float64 {
    return 2*r.width + 2*r.height
}

func (c circle) area() float64 {
    return math.Pi * c.radius * c.radius
}
func (c circle) perim() float64 {
    return 2 * math.Pi * c.radius
}

func measure(g geometry) {
    fmt.Println(g)
    fmt.Println(g.area())
    fmt.Println(g.perim())
}
r := rect{width: 3, height: 4}
c := circle{radius: 5}
measure(r)
measure(c)
\end{lstlisting}

\subsubsection{Skládání rozhraní}
Při vytváření nového rozhraní máme možnost použít již existující rozhraní.

\begin{lstlisting}[caption={Skládání rozhraní}, captionpos=b, language=Golang]
type Interface1 interface {
        method1()
}
 
type Interface2 interface {
        Interface1
        method2()
}
 
type Type struct{}
 
func (Type) method1() {
        fmt.Println("Type.method1")
}
 
func (Type) method2() {
        fmt.Println("Type.method2")
}
 
func f1(i Interface2) {
        i.method1()
        i.method2()
}

t := Type{}
f1(t)

\end{lstlisting}

\subsection{Konstrukce gorutin}
Takzvané \textit{gorutiny} slouží v jazyce Go pro vytváření paralelních programů. Klíčové slovo \texttt{go} je vyhrazené pro spouštění gorutin, které ve zjednodušené formě představují vlákna. Za tímto klíčovým slovem se musí nacházet volání nějaké funkce nebo definice funkce, která se má v nové gorutině spustit.

\begin{lstlisting}[caption={Možnosti spuštění gorutin}, captionpos=b, language=Golang]

func heavyTask() {
    sleep(12 000)
}

go heavyTask() 

go func() { for { sleep(100) }}

\end{lstlisting}

\subsection{Komunikace mezi gorutinami}

Používání kanálů či front pro asynchronní komunikaci mezi různými částmi vyvíjených aplikací se v dnešní době těší poměrně velké popularitě, a proto je tento obecný koncept hlavním prostředkem pro komunikaci mezi gorutinami.

Kanál (implicitní velikosti 1) pro komunikaci mezi gorutinami se vytváří pomocí použití klíčového slova \texttt{chan} následovaným datovým typem, jehož hodnoty budou kanálem přenášeny. Pro vkládání a vyjmutí hodnoty z kanálu se používá operátor \texttt{<-}. Pozice relativní k identifikátoru kanálu určuje, zdali se do kanálu vkládá nebo se z něj vyjímá.

Pokud je kapacita kanálu plná, tak operace pro vklad do kanálu je blokující, a je provedena až tehdy, když se v kanálu uvolní místo. Stejně tomu tak je i v případě, pokud vyjímáme z kanálu, který je prázdný. Operace pro vyjmutí je blokující, dokud do kanálu není vložen nějaký prvek.

\begin{lstlisting}[caption={Využití kanálu}, captionpos=b, language=Golang]
package main
 
import "fmt"
 
func loadValue(id int, out chan int) {
        fmt.Printf("goroutine %d\n", id)
 
        // simulate a delay
        sleep(10000)
        // write into the channel 
        out <- 10
}
 
func main() {
        channel := make(chan int)
 
        go message(1, channel)
 
        fmt.Println("waiting...")
 
        // blocking read 
        value, status := <-channel
 
        fmt.Printf("received value: %d and status: %t\n", value, status)
        fmt.Println("main end")
}


\end{lstlisting}

\subsubsection{Klíčové slovo \texttt{select}}

Pomocí klíčového slova \texttt{select} se definuje množina operací vkládání a vyjímání nad kanálami, které se mají provést. Syntaxe je podobná jako pří používání klíčového slova \texttt{switch}, avšak s tím rozdílem, že za klíčovým slovem \texttt{case} se nachází operace nad kanálem, která se má provést. Pokud je některá operace z množiny provedena, tak je spuštěna její větev.


\begin{lstlisting}[caption={Klíčové slovo \texttt{select}}, captionpos=b, language=Golang]
import (
    "fmt"
    "time"
)
c1 := make(chan string)
c2 := make(chan string)

go func() {
    time.Sleep(1 * time.Second)
    c1 <- "first"
}()
go func() {
    time.Sleep(2 * time.Second)
    c2 <- "second"
}()

for i := 0; i < 2; i++ {
    select {
    case msg1 := <-c1:
        fmt.Println("received", msg1)
    case msg2 := <-c2:
        fmt.Println("received", msg2)
    }
}
\end{lstlisting}

Pokud se přidá větev označená klíčovým slovem \texttt{default}, tak její tělo se provede, pokud operace z definované množiny jsou blokující.

\subsection{Vrácení většího množství hodnot z funkce}

V Go je možné, aby funkce vracela více než pouze jednu hodnotu. Tímto lze omezit zbytečné definování obalovacích struktur (např. struktura o dvou hodnotách).


\begin{lstlisting}[caption={Funkce vracející dvě hodnoty}, captionpos=b, language=Golang]
func swap(a int, b int) (int, int) {
        return b, a
}
 
x := 1
y := 2

z, w := swap(x, y)
fmt.Println("z =", z, "w =", w)
\end{lstlisting}

\subsection{Klíčové slovo \texttt{defer}}

Toto klíčové slovo se používá tehdy, když chceme provést nějaké volání až později při běhu programu. Typicky se jedná o volání, která uvolňují použitou paměť nebo např. uzavírají ukazatele na soubory. Zjednodušeně se dá říct, že se \texttt{defer} používá tam, kde by se v jiných jazycích používalo klíčové slovo \texttt{finally} nebo \texttt{ensure}.



\begin{lstlisting}[caption={Klíčové slovo \texttt{defer}}, captionpos=b, language=Golang]

import (
    "os"
)

func createFile(p string) *os.File {
    f, err := os.Create(p)
    if err != nil {
        panic(err)
    }
    return f
}

func writeFile(f *os.File) {
    fmt.Fprintln(f, "data")
}

func closeFile(f *os.File) {
    err := f.Close()

    if err != nil {
        os.Exit(1)
    }
}

f := createFile("/tmp/defer.txt")
// thanks to defer closeFile will get called after writeFile even if writeFile panics
defer closeFile(f)
writeFile(f)
\end{lstlisting}


\section{Dokumentace a podpora ze strany vývojářů}

Dokumentace je při vývoji softwaru nesmírně důležitá, a prakticky žádný vývojář se bez ní neobejde. Měla by být obsáhlá, ale zároveň i čitelná a precizní. Vývojáři Go si toto uvědomují, a dokumentace tohoto programovacího jazyka je na velice vysoké úrovni\footnote{\url{https://golang.org/doc/}}. K tvorbě dokumentace je dostupný i program \texttt{godoc}, který dokáže generovat dokumentaci přímo z kódu.

Jazyk Go a jeho standardní knihovna je svými vývojáři hojně podporována, kdy jazyk stále aktivně vyvíjí, opravují chyby, naslouchají komunitě, a zabývají se navrhy a připomínkami. Hojná interakce s komunitou je z velké části způsobena tím, že Go je open-source (zrcadlo oficiálního repozitáře \url{https://github.com/golang/go}). 

Za poslední dva roky byly vydány 4 nové \textit{minor} verze - Go se drží půlročního cyklu vydávání nových verzí.

\section{Využití jazyka}

Dle Tiobe indexu je Go 14. nejpoužívanější jazyk\footnote{\url{https://www.tiobe.com/tiobe-index/}}. V ročním přehledu služby GitHub\footnote{\url{https://octoverse.github.com/}} se tento jazyk objevuje v seznamu nejrychleji rostoucích jazyků.

V jazyce Go jsou například napsány v dnešní době hojně používané technologie Docker\footnote{\url{https://www.docker.com/}} a Kubernetes\footnote{\url{https://kubernetes.io/}}. Dále i software pro tvorbu statických webových stránek Hugo\footnote{\url{https://gohugo.io/}} nebo databázový systém InfluxDB\footnote{\url{https://www.influxdata.com/}}.

Nejvyšší využití tedy jazyk Go nachází v software, kde se klade důraz na výkon, a bývá považován za alternativu k jazykům C, C++ nebo Rust.


\section{Vlastní zkušenost}

S jazykem Go mám vlastní zkušenost ze dvou střědně velkých projektů - semestrálních prací z předmětů KIV/UPS a KIV/ZOS.

Při používání tohoto jazyka a jeho ekosystému jsem nejvíc ocenil stručnost, jednoduchost tohoto jazyka a výkon přeložených programů. Vlastností, která je pro mě při výběru programovacího jazyka pro nový projekt velmi důležitá, je jak dobře jeho ekosystém podporuje testování. Narozdíl od jiných nízkoúrovňových jazyků (např. C nebo C++) je podpora testů přímo ve standardní knihovně, a jejich psaní je rychlé a bezproblémové.

Velkou výhodou je i existence speciálního IDE \textbf{GoLand} od společnosti \textbf{JetBrains} pro jazyk Go, která zaručuje rychlý a jednoduchý přechod na používání tohoto
jazyka, pokud již IDE od této společnosti používáte.


\end{document}    
